\section{Обзор метода Гаусса}
\[
\begin{cases}
  a_{11}x_1 + a_{12}x_2 + \cdots + a_{1n}x_n = b_1 \\
  a_{21}x_1 + a_{22}x_2 + \cdots + a_{2n}x_n = b_2 \\
  \dots                                        \\
  a_{n1}x_1 + a_{n2}x_2 + \cdots + a_{nn}x_n = b_n
\end{cases}
\]

\begin{gather*}
  a_{nn}^{(n - 1)} \cdot x_n = b_n^{(n - 1)} \\
  x_n = \frac{b_n^{(n - 1)}}{a_{nn}^{(n - 1)}}
\end{gather*}

\begin{note}
  Процедура выбора ведущего элемента позволяет избежать накопления ошибки в методе Гаусса.
  \begin{enumerate}
  \item Выбор ведущего элемента по столбцу
  \item Выбор ведущего элемента по всей матрице
  \end{enumerate}
\end{note}

\begin{note}
  Оценка числа операций в методе Гаусса 
  \[
    N \sim \frac 23 n^3
  \]
\end{note}

\noindent
Решение вырожденных систем
\[
  \det{A} = 0
\]

\begin{note}[Альтернатива Фредгольма]
  \begin{gather*}
    Rank(A) = Rank(A, \vec{b}) \implies \text{решений бесконечно много} \\
    Rank(A) \neq Rank(A, \vec{b}) \implies \text{решений не существует}
  \end{gather*}
\end{note}

