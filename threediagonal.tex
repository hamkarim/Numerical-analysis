\section{Решение систем алгебраических уравнений с трехдиагональной матрицей. Метод прогонки}

\begin{gather*}
  b_1 x_1 + c_1 x_2  = d_1\\
  a_2 x_1 + b_2 x_2 + c_2 x_3 = d_2 \\
  \dots \\
  a_i x_{i - 1} + b_i x_i + c_i x_{i + 1} = d_i \\
  \dots \\
  a_{n - 1} x_{n - 2} + b_{n - 1} x_{n - 1} + c_{n - 1} x_n = d_{n - 1}
  a_n x_{n - 1}+ b_n x_n = d_n
\end{gather*}

\[
  A =
  \begin{pmatrix}
    b_1      &c_1   &\cdots   &0         &0\\
    a_2      &b_2   &c_2      &\dots     &0\\
    \dots    &a_i   &b_i      &c_i       &\dots\\
    \vdots   &      & \ddots  &          & \vdots\\
    0        &\dots &a_{n - 1}&b_{n - 1} &c_{n - 1}\\
    0        & 0    & \dots   &a_n       &b_n
  \end{pmatrix}
\]

\begin{gather*}
  x_1 = \underbrace{-\frac{c_1}{b_1}}_{\alpha_1} x_2 +
  \underbrace{\frac{d_1}{b_1}}_{\beta_1} \\
  a_2(\alpha_1 x_2 + \beta_1) + \beta_2 x_2 + c_2 x_3 = d_2 \\
  (a_2 \alpha_1 + b_2) x_2 + c_2 x_3 = d_2 - a_2 \beta_1 \\
  x_2 = \underbrace{\frac{c_2}{a_2\alpha_1 + b_2}}_{\alpha_2} + \underbrace{\frac{d_2 - a_2 \beta_1} {a_2 \alpha_1 +
    b_2}}_{\beta_2} = \alpha_2 + x_3 + \beta_2 \\
  \vdots\\
  \begin{cases}
    x_i = \alpha_i x_{i + 1} + \beta_i\\
    \alpha_i = - \frac{c_i}{a_i \alpha_{i - 1} + b_i} \\
    \beta_i = \frac{d_i - a_i \beta_{i - 1}}{a_i \alpha_{i - 1} b_i}
  \end{cases}
\end{gather*}

Сначала прямым проходом находим коэффициенты $\alpha$ и $\beta$, потом обратным
проходом находим неизвестные $x_n, x_{n - 1}, \dotsc$. 

\[
  N \sim 6n
\]

\begin{note}
  \[
    \frac{\|\vec{x}^* - \overline{\vec{x}}\|}{\|\vec{x}\|} \leq \epsilon
  \]
  Если данное условие не выполняется, это называется накоплением ошибки.
\end{note}

\begin{note}
  \[
    |b_k| \geq |a_k| + |c_k|
  \] 
  Достаточное условие ненакопления ошибки.
\end{note}