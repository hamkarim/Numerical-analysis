\section{Численное решение систем нелинейных алгебраических (трансцендентных)
  уравнений}

\begin{gather*}
  \begin{cases}
    f_1(x_1, x_2, \dotsc, x_n) = 0\\
    f_2(x_1, x_2, \dotsc, x_n) = 0\\
    \vdots\\
    f_n(x_1, x_2, \dotsc, x_n) = 0
  \end{cases} \\
\end{gather*}

\begin{enumerate}
\item Предварительный анализ
\item Локализация решений: нахождение $\sigma$-окрестности $\vec{x}$, в которой
  существует единственное решение.
\item Итерационное уточнение решения
\end{enumerate}

\subsection{Метод простых итераций}
\begin{gather*}
  \vec{f}(\vec{x}) = 0 \iff \vec{x} = \vec{\phi}(\vec{x}) \\
  \vec{x}^0. \vec{x}^1 = \vec{\phi}(\vec{x}^0),\dotsc, \vec{x}^{k + 1} =
  \vec{\phi}(\vec{x}^k) \\
  \begin{cases}
    x_1^{k + 1} = \phi_1(z_1^k, \dotsc, x_n^k) \\
    \vdots\\
    x_n^{k + 1} = \phi_n(x_1^k, \dotsc, x_n^k)
  \end{cases}
\end{gather*}

$\sigma$ - окр-ть $\overline{\vec{x}}\ \|\vec{\phi}'(\vec{x})\| \leq q < 1$

\[
  \vec{\phi}'(\vec{x}) = \frac{\partial \vec{\phi}}{\partial \vec{x}} =
  \frac{D(\phi_1, \phi_2, \dotsc, \phi_n)}{D(x_1, x_2, \dotsc, x_n)} =
  \begin{pmatrix}
    \frac{\partial \phi_1}{\partial x_1}
  \end{pmatrix}
\] - $\vec{x}^0 \in \sigma\text{- окр-ть} \overline{\vec{x}} \implies
\|\vec{x}^k - \overline{\vec{x}}\| \leq q^k \|\vec{x}^0 - \overline{\vec{x}}\|$
--- априрорная оценка сходимости.
$x^k \in \sigma\text{- окр-ть } \overline{\vec{x}}$

\[
  \|\vec{x}^k - \overline{\vec{x}}\| \leq \frac{q}{1 - q} \|\vec{X}^k -
  \vec{x}^{k - 1}\| \leq \epsilon \implies \|\vec{x}^k - \vec{x}^{k - 1}\| \leq
  \frac{1 - q}{q} \epsilon
\] --- апостариорная оценка